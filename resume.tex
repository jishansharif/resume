%----------------------------------------------------------------------------------------
%	PACKAGES AND OTHER DOCUMENT CONFIGURATIONS
%----------------------------------------------------------------------------------------
% !TeX spellcheck = en
\documentclass{resume} % Use the custom resume.cls style

\usepackage[left=0.75in,top=0.52in,right=0.75in,bottom=0.52in]{geometry} % Document margins
\usepackage[colorlinks=true, linkcolor=black, urlcolor=black]{hyperref}
\usepackage{marvosym}
\usepackage{graphicx}

\name{{Jishan Sharif}}
\address{\href{mailto:sharifj@mcmaster.ca}{sharifj@mcmaster.ca}\ \\
         \href{https://www.linkedin.com/in/jishan-sharif-37bbb2160/}{in/Jishan Sharif} \\
         \href{https://github.com/jishansharif}{github.com/jishansharif}}

\begin{document}

%----------------------------------------------------------------------------------------
%	SUMMARY SECTION
%----------------------------------------------------------------------------------------

\begin{rSection}{Skills}

  {
    \item Interested in Backend Engineering, Systems Design, Algorithms, and Web technologies.
    \item  Proficient in Go, Python, Haskell, JavaScript , Bash Scripting, HTML.
  }

\end{rSection}

%----------------------------------------------------------------------------------------
%	EDUCATION SECTION
%----------------------------------------------------------------------------------------

\begin{rSection}{Education}
  \begin{rEducationSection}{McMaster University}
                           {Sept '19 -- Present}
                           {Faculty of Engineering in Honours Computer Science (with Distinction)}
  \end{rEducationSection}
\end{rSection}

%----------------------------------------------------------------------------------------
% WORK SECTION
%----------------------------------------------------------------------------------------

\begin{rSection}{Personal Projects}
  \begin{rWorkSection}{Lissajous Design}
                           
    
  {
    \item Created a sequence of bit-mapped images and then encode the sequence as a GIF animation.
    \item Made use of const declaration and creating a struct type.
  }
                           
  \end{rWorkSection}

  \begin{rWorkSection}{Weater Station}
                      
  {
    \item Program desgined to read input from a website and prints the weather conditions for a given location.
    \item Made use of an API call which allowed access to internet resources when formatted by json.
       
     }
  \end{rWorkSection}

  \begin{rWorkSection}{Arithmetic Calculator}
                     
                     
                     
  {
    \item Created an interactive calculator allowing users to perform calculations in real time.
    \item Made use of a stack call, values entered by the user is stored and is called only during runtime,
    
       }
  \end{rWorkSection}

  \begin{rWorkSection} {Word Counter Program}
                     
                    
  {
    \item Program designed to read data from a text file, and outputs the most frequent word used. 
    \item Made use of a dictionary to store the word used and it's frequency. This dictionary was then converted into a sorted list.
       
       }

  \end{rWorkSection}


\begin{rWorkSection} {TODO List}
                     

{
\item Program desgined to show the tasks to be completed by a specific date, and it's respected priority.
\item Designed the program using a struct type declaration allowing the program to be readable and consise.
    }

\end{rSection}

\begin{rWorkSection} {Transliteration Program}

  {
  \item The purpose of this dictionary is to specify a language and a word or phrase, and get as a result the translation of that word or phrase in the specified language. 
  \item The first key will be the language, the second the word to be translated.This program required use of nested dictionaries.
  
  }
\end{rWorkSection}

\begin{rWorkSection} {CheckSum}
  {
  \item Checksum is a simple but effective way to verify the integrity of transmitted data.  
  \item Given a list of data and a checksum value, return true or false, based on whether the data matches the checksum value.
  \item This program could be used for cryptography and ensuring data security.
  }
\end{rWorkSection}






%----------------------------------------------------------------------------------------
% RESEARCH SECTION
%----------------------------------------------------------------------------------------


%----------------------------------------------------------------------------------------
%	PROJECTS SECTION
%----------------------------------------------------------------------------------------

\begin{rSection}{Volunteer}
  \begin{rProjectSection}
    \item Community Service helping special need students over the start of 2017. 
    \item Been a private tutor and have taught Advanced Physics and Mathematics from 2017 till September 2019.
    \item Partnered with Al Noor Dubai to help Differently abled students. Took active participation in fundraisers to help provide services for children with special needs.

    

  \begin{rBlurbSection}
  
\end{rSection}

\end{document}
